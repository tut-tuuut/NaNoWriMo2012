\section{pouf_pouf}
Le vent de novembre était froid et mordant. Je sortais de la cantine avec Jennifer, ma meilleure copine à la vie à la mort. La fac était aussi grise que d'habitude, l'après-midi promettait d'être aussi morne que d'habitude. La vie semblait aussi normale que d'habitude, et je me sentais assez prompte à me répéter dans ma tête.

\bnw{Comment allais-je passer la suite de ma pause midi ?}
\item J'ai entendu parler d'un musée des horreurs à la fac. Je vais aller y faire un petit tour. \go{horreurs}
\item J'ai plutôt envie de rester tranquillement sur la pelouse, à profiter du soleil, de la douce brise… hmm… bon, j'ai juste envie de rester tranquillement quelque part. En plus, il y a des petits écureuils qui sautent de branche en branche dans le petit bois à côté de la cour. C'est mignon. \go{ecureuils}
\enw

\section{horreurs}

Le musée des horreurs n'est pas vraiment un musée des horreurs. C'est le musée de l'histoire de la médecine. Mais comme il est plein de créatures malformées, les étudiants ont vite fait de l'appeler « le musée des horreurs ». Ils manquent peut-être un peu d'imagination…

Je me dirigeai, avec Jennifer à mes côtés, vers le petit bâtiment accolé au bâtiment principal, dans lequel je sais que le petit musée des horreurs montre ses charmes à qui veut les voir. Nous commençons à vagabonder le long des allées. Ici, un foetus à deux têtes, là, une chèvre à un oeil… ou l'inverse. Je me sens mal au bout de quelques plaques indicatrices.

\begin{itemize}
\item Jen, je me sens mal. Je n'aime pas cet endroit.
\item Allez, Pen (oui, c'est mon surnom). Viens, il y a encore des choses amusantes de ce côté… Oh, regarde ce petit bocal avec un bébé chat dedans…
\item Jen, tu as un humour très glauque. Tu ne me mets pas à l'aise, tu sais.
\item Fais pas ta mijaurée. Tu as voulu faire médecine, tu assumes.
\item J'ai fait médecine, pas Dr. Frankenstein.
\end{itemize}

Jennifer, et ses deux couettes brunes, s'éloigne dans une allée flanquée de deux rangées de bocaux pleins de bestioles de cauchemar. Je la suis, en essayant de ne pas trop prêter attention aux regards morts qui me fixent depuis leurs cages de verre et de formol.

Je réprime un frisson en tentant de me donner une certaine contenance. Soudain, il me semble voir une tête coupée, flottant dans un bocal, me lancer quelque chose comme… un clin d'œil ?

\bnw{et maintenant…}
\item J'hallucine vraiment. J'appelle Jen pour que l'on quitte cet endroit. J'ai dû manger quelque chose de pas très frais à midi.
\item Intéressant. Je m'approche pour observer cette tête de plus près.
\enw
