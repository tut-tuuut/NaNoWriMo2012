\section{poufpouf}
Le vent de novembre était froid et mordant. Je sortais de la cantine avec Jennifer, ma meilleure copine à la vie à la mort. La fac était aussi grise que d'habitude, l'après-midi promettait d'être aussi morne que d'habitude. La vie semblait aussi normale que d'habitude, et je me sentais assez prompte à me répéter dans ma tête.

\bnw{Comment allais-je passer la suite de ma pause midi ?}
\item J'ai entendu parler d'un musée des horreurs à la fac. Je vais aller y faire un petit tour. \go{horreurs}
\item J'ai plutôt envie de rester tranquillement sur la pelouse, à profiter du soleil, de la douce brise… hmm… bon, j'ai juste envie de rester tranquillement quelque part. En plus, il y a des petits écureuils qui sautent de branche en branche dans le petit bois à côté de la cour. C'est mignon. \go{ecureuils}
\enw

\section{horreurs}

Le musée des horreurs n'est pas vraiment un musée des horreurs. C'est le musée de l'histoire de la médecine. Mais comme il est plein de créatures malformées, les étudiants ont vite fait de l'appeler « le musée des horreurs ». Ils manquent peut-être un peu d'imagination…

Je me dirigeai, avec Jennifer à mes côtés, vers le petit bâtiment accolé au bâtiment principal, dans lequel je sais que le petit musée des horreurs montre ses charmes à qui veut les voir. Nous commençons à vagabonder le long des allées, entre les étagères chargées de bocaux pleins de spectres inquiétants. Ici, un fœtus à deux têtes, là, une chèvre à un œil… ou l'inverse. Je me sens mal au bout de quelques plaques indicatrices.

\begin{itemize}
\item Jen, je me sens mal. Je n'aime pas cet endroit.
\item Allez, Pen (oui, c'est mon surnom). Viens, il y a encore des choses amusantes de ce côté… Oh, regarde ce petit bocal avec un bébé chat dedans…
\item Jen, tu as un humour très glauque. Tu ne me mets pas à l'aise, tu sais.
\item Fais pas ta mijaurée. Tu as voulu faire médecine, tu assumes.
\item J'ai fait médecine, pas Dr. Frankenstein.
\end{itemize}

Jennifer, et ses deux couettes brunes, s'éloigne dans une allée flanquée de deux rangées de bocaux pleins de bestioles de cauchemar. Je la suis, en essayant de ne pas trop prêter attention aux regards morts qui me fixent depuis leurs cages de verre et de formol.

Je réprime un frisson en tentant de me donner une certaine contenance. Soudain, il me semble voir une tête coupée, flottant dans un bocal, me lancer quelque chose comme… un clin d'œil ?

\bnw{Une tête en bocal qui fait des clins d'œil ? Rien que ça ? Et maintenant ?}
\item J'hallucine vraiment. J'appelle Jen pour que l'on quitte cet endroit. J'ai dû manger quelque chose de pas très frais à midi.
\item Intéressant. Je m'approche pour observer cette tête de plus près. \go{tete-en-bocal}
\enw

\section{tete-en-bocal}

Je jette un regard angoissé dans la direction où Jen s'est éloignée, mais je ne la vois pas. J'entends ses ricanements dans les allées de l'autre côté des petits cadavres en conserve, cependant. Je reporte mon attention vers le bocal contenant la tête. Sa peau est un peu pourrie. Elle arbore des reflets verdâtres sur un fond gris clair. Ses yeux sont à moitié fermés, j'entrevois des blancs des yeux un peu jaunâtres, et la bouche laisse apparaître quelques dents brunâtres entre deux lèvres bleutées.

Elle ne bouge pas. Elle garde son air somnolent, bouche et yeux entrouverts, dans son bocal de formol. Elle flotte tranquillement, comme une tête de cadavre détachée de son corps. Évidemment, j'ai halluciné. Je vais me méfier de la cantine, à l'avenir.

Je m'approche, pour réussir à lire le minuscule texte explicatif disposé juste devant le bocal.

\textit{Zombixnitus machinus bidulus galti—}.

Soudain, un mouvement au coin de mon champ de vision me fait sursauter. La bouche de la tête vient de s'ouvrir brusquement, en grand, révélant des chicots brun-noir plus effrayants encore que lorsqu'ils n'étaient que suggérés derrière les lèvres violacées. Mon sang ne fait qu'un tour et mon cœur rate un battement. Je me relève brusquement et je bute dans Jen, qui s'était approchée silencieusement derrière moi.

\begin{itemize}
\item Hé, tu m'as fait peur !, grogné-je.
\item Oah, ne t'inquiète pas. Tu avais peur de la petite tête dans son petit bocal ?
\item Je n'avais pas peur de cette petite tête. J'avais peur de tout le reste. L'ambiance ici me pèse, je me casse.
\end{itemize}

Jen est une copine super par moments, mais il y a aussi des moments où elle me fatigue beaucoup. Aujourd'hui, elle est d'ailleurs particulièrement cynique, et je ne la supporte pas. Je m'apprête à décoller, je commence à m'éloigner, quand j'entends la voix de Jen faire l'écho de mes propres pensées quelques minutes auparavant.

\begin{itemize}
\item Oh merde. Elle a bougé.
\item Jen, dis-je. Tu hallucines. J'ai cru que cette tête bougeait aussi, tout à l'heure, mais je t'assure que les têtes en bocal ne boug…
\end{itemize}

Je n'ai pas le temps de finir ma phrase. Un claquement sec se fait entendre. Je me retourne pour voir que le verre du bocal contenant la tête s'est fendillé, et que la tête, qui a maintenant les yeux grands ouverts, semble avoir commencé à se dissoudre dans le liquide.

\begin{itemize}
\item Jen, qu'est-ce qui se passe ? Qu'es-ce que tu as fait à cette tête ?
\item J'ai rien fait, Pen ! Le truc a ouvert les yeux d'un coup et le verre a claqué à ce moment-là. Je…
\end{itemize}

Une explosion se fait entendre et tout devient noir.

\bnw{Et maintenant}
\item Rendez-vous au paragraphe \go{tout-est-noir}
\enw

\section{tout-est-noir}

Tout est noir. La douleur est terrible.

Le visage me brûle, j'ai mal dans tous mes membres. Comme des centaines de milliers de petits poignards qui me transpercent la peau et les os. Je souffre.

Progressivement, j'arrive à distinguer les choses autour de moi. Je vois un corps ensanglanté à côté de moi, étendu au milieu des éclats de verre d'épaisseur impressionnante. Si nous avons été frappées par cela, nul doute que nous ne sommes pas belles à voir.

\begin{itemize}
\item Jen… tu m'entends ? articulé-je péniblement.
\end{itemize}

Jen remue faiblement à côté de moi. Je remue un bras pour le tendre vers mon amie. La douleur qui me transperce aussitôt est insoutenable. Je manque de retomber dans l'inconscience, mais je m'accroche.

Et soudain, la douleur reflue, je me sens mieux, je vois plus clair autour de moi.

Les endorphines, c'est puissant. On a vu ça en cours. Mais je ne pensais pas que c'était puissant à ce point là.

Je profite de cet instant de répit pour me remettre debout. Jen remue faiblement puis plus fermement, et elle se remet debout elle aussi. Elle a le visage et le torse en sang, une vilaine balafre lui creuse un gigantesque sourire macabre autour de la bouche. Un peu comme le joker dans Batman.

Je me mets en route vers la sortie. Il nous faut des secours.

\begin{itemize}
\item À l'aide ! Il y a quelqu'un ?
\end{itemize}

Je titube vers la sortie de secours. À chaque pas, j'ai moins mal. Je me sens de moins en moins sensible à la douleur. Une étrange sensation de puissance monte en moi. Un sourire se dessine sur mes lèvres. Et en même temps, j'entends Jen ricaner à ma droite, d'un rire de soulagement. Ouf, si on est debout et qu'on ne souffre plus, ce n'est pas si grave.

Un homme entre dans la pièce devant moi. Je reconnais monsieur Tartempion, le concierge, que j'ai croisé souvent pendant les pauses. Il ouvre de grands yeux surpris en m'apercevant.

J'ouvre la bouche pour lui demander de l'aide, mais là, curieusement, je me mets en colère. Une colère pure, de la fureur brûlante. Il doit payer. Cet innocent concierge doit payer pour tous les autres.

Je pousse un hurlement de fureur mêlé d'un grognement coléreux. Je me jette sur le pauvre monsieur Tartempion, qui se laisse surprendre, et je le mords à l'épaule de toutes mes forces. Jen se joint à moi. Je suis furieuse contre lui. Nous le jetons à terre. À coups de mâchoires et de mouvements de tête, nous nous partageons des parts de butin. Une part de moi fait mine de protester, disant quelque part que ce n'est pas très logique, que les gens civilisés n'en attaquent pas d'autres sans une bonne raison… et pas avec les dents, a priori. mais je suis tellement soulagée d'avoir fait payer monsieur Tartempion que cette dernière étincelle de lucidité m'échappe et disparaît alors que je souris à Jen.

Je tords le cou à la petite voix qui me murmure que tout cela n'est pas COMPLÈTEMENT normal, puis je pars à la suite de Jen pour chercher un peu plus de nourriture.

\go{chercher-de-la-nourriture}

\section{chercher-de-la-nourriture}

Miam miam. J'ai faim. J'ai envie de faire souffrir des humains.

Les couloirs à côté du musée des horreurs sont déserts. Je me tourne vers Jen pour lui dire que nous allons continuer vers le bâtiment principal. Mais quand je parle, tout ce que j'arrive à émettre comme bruit, c'est :

\begin{itemize}
\item Hreu greumf grr aaah uuuh.
\end{itemize}

Jen, quant à elle, me répond :

\begin{itemize}
\item Grra aaah uuh rrrhhaa haa huu.
\end{itemize}

Je crois qu'on s'est comprises. Nous nous éloignons du musée des horreurs, en boitillant légèrement et en laissant une impressionnante traînée de sang derrière nous, vers le bâtiment où nous entendons que plein de vie palpite dans l'attente de nos claquements de dents.

Nous poussons une porte coupe-feu. De l'autre côté, nous croisons deux étudiants pleins de vie. Je leur souris. Je les déteste, mais je suis contente de les voir.

Je lis la peur dans leurs yeux. Ils esquissent un mouvement de recul, mais je suis plus rapide. Enfin, j'essaie. J'ai du mal à coordonner mes mouvements. Je m'emmêle légèrement les pieds et je trébuche, et je tombe à la renverse dans un gracieux « floc » humide. Je ne suis peut-être pas dans un état aussi \textit{normal} que cela après tout.

\bnw{Étalée par terre, que vais-je faire maintenant ?}
\item Tout cela est grave. Je tends la main vers eux pour leur demander de l'aide. \go{demander-de-laide}
\item Je vais me relever, et je vais les attraper pour les manger, pardi. \go{manger-les-etudiants}
\enw

\section{demander-de-laide}

\begin{itemize}
\item Hurr rgrr rrahh rrhuuh.
\end{itemize}

C'est tout ce que j'arrive à dire. Je crois que la communication va être compliquée avec les humains.

Mes deux proies et leurs cerveaux si appétissants se sauvent en courant. Attendez ! Je ne veux pas… en fait si, je veux. Je veux vous mordre jusqu'au sang, vous dévorer, vous faire payer.

\textit{Mais… vous faire payer quoi ?}

Cette petite voix raisonnable est insupportable, décidément. Je me relève en poussant péniblement sur mes bras ensanglantés. Jen me dépasse pour se lancer à la suite des deux étudiants infortunés. Je me lance à sa suite. Ces deux bouts de viande ne vont pas nous échapper comme cela.

\section{manger-les-etudiants} 

Je ne vais pas laisser une chute si anodine m'empêcher de savourer un bon repas plein de viande crue encore chaude.

\textit{Mais… pourquoi de la viande crue ?}