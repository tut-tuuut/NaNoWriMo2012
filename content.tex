\section{poufpouf}
Le vent de novembre était froid et mordant. Je sortais de la cantine avec Jennifer, ma meilleure copine à la vie à la mort. La fac était aussi grise que d'habitude, l'après-midi promettait d'être aussi morne que d'habitude. La vie semblait aussi normale que d'habitude, et je me sentais assez prompte à me répéter dans ma tête.

\bnw{Comment allais-je passer la suite de ma pause midi ?}
\item J'ai entendu parler d'un musée des horreurs à la fac. Je vais aller y faire un petit tour. \go{horreurs}
\item J'ai plutôt envie de rester tranquillement sur la pelouse, à profiter du soleil, de la douce brise… hmm… bon, j'ai juste envie de rester tranquillement quelque part. En plus, il y a des petits écureuils qui sautent de branche en branche dans le petit bois à côté de la cour. C'est mignon. \go{ecureuils}
\enw

\section{horreurs}

Le musée des horreurs n'est pas vraiment un musée des horreurs. C'est le musée de l'histoire de la médecine. Mais comme il est plein de créatures malformées, les étudiants ont vite fait de l'appeler « le musée des horreurs ». Ils manquent peut-être un peu d'imagination…

Je me dirigeai, avec Jennifer à mes côtés, vers le petit bâtiment accolé au bâtiment principal, dans lequel je sais que le petit musée des horreurs montre ses charmes à qui veut les voir. Nous commençons à vagabonder le long des allées, entre les étagères chargées de bocaux pleins de spectres inquiétants. Ici, un fœtus à deux têtes, là, une chèvre à un œil… ou l'inverse. Je me sens mal au bout de quelques plaques indicatrices.

\begin{itemize}
\item Jen, je me sens mal. Je n'aime pas cet endroit.
\item Allez, Pen (oui, c'est mon surnom). Viens, il y a encore des choses amusantes de ce côté… Oh, regarde ce petit bocal avec un bébé chat dedans…
\item Jen, tu as un humour très glauque. Tu ne me mets pas à l'aise, tu sais.
\item Fais pas ta mijaurée. Tu as voulu faire médecine, tu assumes.
\item J'ai fait médecine, pas Dr. Frankenstein.
\end{itemize}

Jennifer, et ses deux couettes brunes, s'éloigne dans une allée flanquée de deux rangées de bocaux pleins de bestioles de cauchemar. Je la suis, en essayant de ne pas trop prêter attention aux regards morts qui me fixent depuis leurs cages de verre et de formol.

Je réprime un frisson en tentant de me donner une certaine contenance. Soudain, il me semble voir une tête coupée, flottant dans un bocal, me lancer quelque chose comme… un clin d'œil ?

\bnw{Une tête en bocal qui fait des clins d'œil ? Rien que ça ? Et maintenant ?}
\item J'hallucine vraiment. J'appelle Jen pour que l'on quitte cet endroit. J'ai dû manger quelque chose de pas très frais à midi.
\item Intéressant. Je m'approche pour observer cette tête de plus près. \go{tete-en-bocal}
\enw

\section{tete-en-bocal}

Je jette un regard angoissé dans la direction où Jen s'est éloignée, mais je ne la vois pas. J'entends ses ricanements dans les allées de l'autre côté des petits cadavres en conserve, cependant. Je reporte mon attention vers le bocal contenant la tête. Sa peau est un peu pourrie. Elle arbore des reflets verdâtres sur un fond gris clair. Ses yeux sont à moitié fermés, j'entrevois des blancs des yeux un peu jaunâtres, et la bouche laisse apparaître quelques dents brunâtres entre deux lèvres bleutées.

Elle ne bouge pas. Elle garde son air somnolent, bouche et yeux entrouverts, dans son bocal de formol. Elle flotte tranquillement, comme une tête de cadavre détachée de son corps. Évidemment, j'ai halluciné. Je vais me méfier de la cantine, à l'avenir.

Je m'approche, pour réussir à lire le minuscule texte explicatif disposé juste devant le bocal.

\textit{Zombixnitus machinus bidulus galti—}.

Soudain, un mouvement au coin de mon champ de vision me fait sursauter. La bouche de la tête vient de s'ouvrir brusquement, en grand, révélant des chicots brun-noir plus effrayants encore que lorsqu'ils n'étaient que suggérés derrière les lèvres violacées. Mon sang ne fait qu'un tour et mon cœur rate un battement. Je me relève brusquement et je bute dans Jen, qui s'était approchée silencieusement derrière moi.

\item Hé, tu m'as fait peur !, grogné-je.
\item Oah, ne t'inquiète pas. Tu avais peur de la petite tête dans son petit bocal ?
\item Je n'avais pas peur de cette petite tête. J'avais peur de tout le reste. L'ambiance ici me pèse, je me casse.

Jen est une copine super par moments, mais il y a aussi des moments où elle me fatigue beaucoup. Aujourd'hui, elle est d'ailleurs particulièrement cynique, et je ne la supporte pas. Je m'apprête à décoller, je commence à m'éloigner, quand j'entends la voix de Jen faire l'écho de mes propres pensées quelques minutes auparavant.
 
\item Oh merde. Elle a bougé.
\item Jen, dis-je. Tu hallucines. J'ai cru que cette tête bougeait aussi, tout à l'heure, mais je t'assure que les têtes en bocal ne boug…

Je n'ai pas le temps de finir ma phrase. Un claquement sec se fait entendre. Je me retourne pour voir que le verre du bocal contenant la tête s'est fendillé, et que la tête, qui a maintenant les yeux grands ouverts, semble avoir commencé à se dissoudre dans le liquide.

\item Jen, qu'est-ce qui se passe ? Qu'es-ce que tu as fait à cette tête ?
\item J'ai rien fait, Pen ! Le truc a ouvert les yeux d'un coup et le verre a claqué à ce moment-là. Je…

Une explosion se fait entendre et tout devient noir.

\bnw{Et maintenant}
\item Rendez-vous au paragraphe \go{tout-est-noir}
\enw

\section{tout-est-noir}

Tout est noir. La douleur est terrible.

Le visage me brûle, j'ai mal dans tous mes membres. Comme des centaines de milliers de petits poignards qui me transpercent la peau et les os. Je souffre.

Progressivement, j'arrive à distinguer les choses autour de moi. Je vois un corps ensanglanté à côté de moi, étendu au milieu des éclats de verre d'épaisseur impressionnante. Si nous avons été frappées par cela, nul doute que nous ne sommes pas belles à voir.

\item Jen… tu m'entends ? articulé-je péniblement.

Jen remue faiblement à côté de moi. Je remue un bras pour le tendre vers mon amie. La douleur qui me transperce aussitôt est insoutenable. Je manque de retomber dans l'inconscience, mais je m'accroche.

Et soudain, la douleur reflue, je me sens mieux, je vois plus clair autour de moi.

Les endorphines, c'est puissant. On a vu ça en cours. Mais je ne pensais pas que c'était puissant à ce point là.

Je profite de cet instant de répit pour me remettre debout. Jen remue faiblement puis plus fermement, et elle se remet debout elle aussi. Elle a le visage en sang, une vilaine balafre lui creuse un gigantesque sourire macabre autour de la bouche. Un peu comme le joker dans Batman.