\section{poufpouf}
Le vent de novembre était froid et mordant. Je sortais de la cantine avec Jennifer, ma meilleure copine à la vie à la mort. La fac était aussi grise que d'habitude, l'après-midi promettait d'être aussi morne que d'habitude. La vie semblait aussi normale que d'habitude, et je me sentais assez prompte à me répéter dans ma tête.

\bnw{Comment allais-je passer la suite de ma pause midi ?}
\item J'ai entendu parler d'un musée des horreurs à la fac. Je vais aller y faire un petit tour. \go{horreurs}
\item J'ai plutôt envie de rester tranquillement sur la pelouse, à profiter du soleil, de la douce brise… hmm… bon, j'ai juste envie de rester tranquillement quelque part. En plus, il y a des petits écureuils qui sautent de branche en branche dans le petit bois à côté de la cour. C'est mignon. \go{ecureuils}
\enw

% ===============================================================
% Portable
% ===============================================================

\section{ecureuils}

Il y a vraiment des écureuils dans les arbres ! Je croyais qu'ils étaient tous partis dormir, avec les premiers frimas de l'hiver.

Délaissant la pelouse, j'indique un banc presque propre à Jennifer. Autant éviter de poser nos fesses dans la boue alors qu'il nous reste plus de quatre heures de cours d'ici ce soir.

Nous nous asseyons, et contemplons les arbres en silence. Le soleil de ce début d'hiver illumine l'air d'une clarté dorée très agréable à l'oeil. Je n'ai jamais été douée pour la poésie ni pour l'art, donc vous devrez vous contenter de cette affirmation : c'est très joli.

Le portable de Jen émet quelques notes cristallines. Elle sort son téléphone de sa poche, lit le SMS qui vient d'arriver et fronce les sourcils. Avec ses couettes brunes, presque noires, et ses lunettes à la montre épaisse, elle a l'air d'une grande gamine contrariée. Je la vois pianoter une réponse, frénétiquement.

\begin{itemize}
\item Que se passe-t-il, Jen ? lui demandé-je. Tu as l'air contrariée.
\item Ismaël essaie de m'éduquer. Apparemment, je n'ai pas fait la vaisselle comme il fallait hier soir.
\end{itemize}

Je ne peux retenir une moue de mépris. Ismaël, le coloc de Jen, est un type arrogant qui a un avis sur tout. Et surtout sur moi.

\begin{itemize}
\item Et alors ? rétorqué-je. Tu as fait la vaisselle. S'il n'est pas content, il n'a qu'à la faire lui-même.
\item C'est un peu plus compliqué que ça, Pénélope. Quand on vit à plusieurs, on doit faire des efforts pour les autres.
\item Mouais. Enfin j'ai plutôt l'impression qu'on doit faire des efforts pour \textit{lui}.
\end{itemize}

Jen ne répond pas. Elle lève les yeux au ciel avec un soupir. Elle sait que ça n'a jamais été le grand amour entre son coloc et moi, mais elle s'obstine à essayer de nous faire nous entendre. C'est pas comme si on allait devoir sauver le monde ensemble, de toute façon…

\begin{itemize}
\item Laisse tomber, dit-elle.
\end{itemize}

Je laisse le silence s'installer entre nous. Jen garde en main son téléphone et commence à relever ses tweets. Les écureuils continuent de sauter d'arbre en arbre au-dessus de la pelouse. J'en ai assez d'entendre parler de son coloc. J'en ai assez que Jen me reproche de sortir avec Ulysse. D'habitude, je m'entends bien avec Jen, c'est ma meilleure copine depuis longtemps, mais ces derniers temps nous passons notre temps à nous prendre le chou sur des broutilles. Je ne sais pas si c'est la météo ou les hormones, ou quoi que ce soit, mais bon.

J'entends comme un claquement étrange du côté du musée des horreurs, où j'ai failli vouloir aller tout à l'heure. Jen l'a entendu aussi, je la vois relever imperceptiblement les yeux de son twitter. Elle ne semble pas s'en inquiéter plus que ça. Le temps s'écoule encore quelques minutes, silence entre nous, les petits bruits de la vie qui rythment le reste du monde autour de nous.

\begin{itemize}
\item Quelle heure est-il ? demande Jen, rompant le silence. On devrait bientôt retourner en cours, je crois.
\end{itemize}

\bnw{Que vais-je lui répondre ?}
\item « Oui, allons-y. Retournons en cours… » \go{retour-en-cours}
\item « J'ai pas tellement envie d'aller en cours cet aprèm… Tu pourrais répondre pour moi ? » \go{ecole-buissonniere}
\enw

\section{retour-en-cours}



\section{ecole-buissonniere}

Jen lève un regard surpris vers moi.

\begin{itemize}
\item Pen, ça va pas ? Tu te sens mal ? Tu vas \textit{vraiment} sécher ?
\item Ben ouais. Je sais pas, j'ai pas la motiv, là.
\item Mais tu es la première à me pousser à bouger même quand j'ai la motiv au trente-sixième dessous. Et en une première année et demi, tu n'as jamais même séché un demi amphi.
\item Je t'ai dit, je sais pas.
\item Bon. Tu es une sale peste, mais je tiens à toi. Alors OK, je te laisse sécher pour cette fois… Mais tu me promets de prendre soin de toi cet après-midi, hmm ?
\item Promis, Jen.
\end{itemize}

On se claque une petite bise, je la regarde s'éloigner vers le grand amphi en traînant son gros sac de cours.

\bnw{C'est bien beau de sécher l'école, mais… qu'est-ce que je vais faire cet après-midi ?}
\item Un petit tour au gymnase, pour m'entraîner au tir à l'arc. \go{gymnase-tir-a-l-arc}
\item Je vais me rentrer chez moi et faire une petite sieste. \go{une-petite-sieste}
\enw

\section{une-petite-sieste}

Je remonte la pente de l'avenue Rockefeller pour attraper un tram. Quelques minutes d'attente plus tard, je suis assise dans un tram et je regarde défiler le paysage. Je me perds rapidement dans mes pensées… Qu'est-ce qui m'a poussée à laisser tomber les cours cet après-midi ? Je me sentais déprimée ? Il faudrait vraiment que je remette les points sur les I avec Jen un de ces jours. Je ne sais pas pourquoi elle est tellement cynique avec moi ces temps-ci… Et malgré ce que je lui laisse croire la plupart du temps, son attitude me pèse.

Alors que le tram atteint la station \textit{Les Alizés}, mon téléphone vibre dans ma poche. Je l'extripe de ma veste en essayant de ne pas laisser tomber mon sac par terre. C'est un SMS de Jen.

\textit{Pb à la fac C dangereux ici reviens pas}

L'absence de ponctuation et la présence d'abréviations douteuses m'inquiète plus que le contenu du message en lui-même. Si le problème qu'il y a là-bas a empêché Jen, la plus grande psychorigide de l'orthographe que je connaisse, de prendre le temps de typographier correctement son SMS, c'est qu'il s'agit d'un problème grave.

Je m'empresse de lui demander plus de détails.

\textit{Keski spasse ? C grave ?}

La réponse arrive instantanément ou presque.

\textit{Gens fous mordent les autres. Contagieux. Sang partout. Grave. Sais pas ce que c'est.}

Je me mords l'intérieur des joues. Mon arrêt est là. Je descends du tram. Le soleil brillant qui perce les feuilles vertes ne laisse pas deviner le « problème » qui se trame un peu plus loin.

\textit{Tu vas bien ?}, textoté-je.

Sa réponse n'arrive pas tout de suite. Je me dirige vers mon immeuble en gardant un œil anxieux sur l'écran de mon téléphone. J'ai le temps de monter jusqu'à mon premier étage avant que la réponse de mon amie ne fasse à nouveau sonner mon téléphone.

\textit{À peu près. Je me cache.}

À peu près ? Qu'est-ce que ça veut dire, à peu près ? Merde ! Jen, qu'est-ce qu'il t'arrive ?

J'entre dans mon appartement. Je ferme la porte derrière moi, je pose mon sac sur la table et je pose les mains à plat sur mon bureau. J'inspire longuement, j'expire, une fois, deux fois.

\bnw{Qu'est-ce que je vais faire maintenant ?}
\item Une petite sieste, comme j'avais dit. \go{dormir}
\item Je prends mon arc et je retourne à la fac. \go{retour-fac-avec-arc}
\item Je regarde si les infos disent quelque chose sur cet incident. \go{voir-les-infos}
\enw

\section{gymnase-tir-a-l-arc}

J'accroche mon gros sac de cours en bandoulière sur mon épaule, et je me dirige vers le gymnase Laënnec. Il faut que je monte la pente de l'avenue Rockefeller, puis que je tourne à droite en haut de la butte. Alors que je remonte la pente, un tram me dépasse dans un bruit de frottement métallique.

Arrivée devant le petit magasin bio du milieu de la rue, je m'arrête. Quelle crétine ! J'allais m'entraîner au tir à l'arc sans mon arc… Celui-ci est chez moi, bien rangé dans sa sacoche.

\bnw{Du coup…}
\item Je n'ai pas le choix, il va bien falloir que je rentre chez moi. \go{une-petite-sieste}
\item Bon bah je vais retourner en cours, même si je vais avoir un peu de retard. \go{retour-en-cours}
\enw

\section{horreurs}

Le musée des horreurs n'est pas vraiment un musée des horreurs. C'est le musée de l'histoire de la médecine. Mais comme il est plein de créatures malformées, les étudiants ont vite fait de l'appeler « le musée des horreurs ». Ils manquent peut-être un peu d'imagination…

Je me dirigeai, avec Jennifer à mes côtés, vers le petit bâtiment accolé au bâtiment principal, dans lequel je sais que le petit musée des horreurs montre ses charmes à qui veut les voir. Nous commençons à vagabonder le long des allées, entre les étagères chargées de bocaux pleins de spectres inquiétants. Ici, un fœtus à deux têtes, là, une chèvre à un œil… ou l'inverse. Je me sens mal au bout de quelques plaques indicatrices.

\begin{itemize}
\item Jen, je me sens mal. Je n'aime pas cet endroit.
\item Allez, Pen (oui, c'est mon surnom). Viens, il y a encore des choses amusantes de ce côté… Oh, regarde ce petit bocal avec un bébé chat dedans…
\item Jen, tu as un humour très glauque. Tu ne me mets pas à l'aise, tu sais.
\item Fais pas ta mijaurée. Tu as voulu faire médecine, tu assumes.
\item J'ai fait médecine, pas Dr. Frankenstein.
\end{itemize}

Jennifer, et ses deux couettes brunes, s'éloigne dans une allée flanquée de deux rangées de bocaux pleins de bestioles de cauchemar. Je la suis, en essayant de ne pas trop prêter attention aux regards morts qui me fixent depuis leurs cages de verre et de formol.

Je réprime un frisson en tentant de me donner une certaine contenance. Soudain, il me semble voir une tête coupée, flottant dans un bocal, me lancer quelque chose comme… un clin d'œil ?

\bnw{Une tête en bocal qui fait des clins d'œil ? Rien que ça ? Et maintenant ?}
\item J'hallucine vraiment. J'appelle Jen pour que l'on quitte cet endroit. J'ai dû manger quelque chose de pas très frais à midi.
\item Intéressant. Je m'approche pour observer cette tête de plus près. \go{tete-en-bocal}
\enw

\section{tete-en-bocal}

Je jette un regard angoissé dans la direction où Jen s'est éloignée, mais je ne la vois pas. J'entends ses ricanements dans les allées de l'autre côté des petits cadavres en conserve, cependant. Je reporte mon attention vers le bocal contenant la tête. Sa peau est un peu pourrie. Elle arbore des reflets verdâtres sur un fond gris clair. Ses yeux sont à moitié fermés, j'entrevois des blancs des yeux un peu jaunâtres, et la bouche laisse apparaître quelques dents brunâtres entre deux lèvres bleutées.

Elle ne bouge pas. Elle garde son air somnolent, bouche et yeux entrouverts, dans son bocal de formol. Elle flotte tranquillement, comme une tête de cadavre détachée de son corps. Évidemment, j'ai halluciné. Je vais me méfier de la cantine, à l'avenir.

Je m'approche, pour réussir à lire le minuscule texte explicatif disposé juste devant le bocal.

\textit{Zombixnitus machinus bidulus galti—}.

Soudain, un mouvement au coin de mon champ de vision me fait sursauter. La bouche de la tête vient de s'ouvrir brusquement, en grand, révélant des chicots brun-noir plus effrayants encore que lorsqu'ils n'étaient que suggérés derrière les lèvres violacées. Mon sang ne fait qu'un tour et mon cœur rate un battement. Je me relève brusquement et je bute dans Jen, qui s'était approchée silencieusement derrière moi.

\begin{itemize}
\item Hé, tu m'as fait peur !, grogné-je.
\item Oah, ne t'inquiète pas. Tu avais peur de la petite tête dans son petit bocal ?
\item Je n'avais pas peur de cette petite tête. J'avais peur de tout le reste. L'ambiance ici me pèse, je me casse.
\end{itemize}

Jen est une copine super par moments, mais il y a aussi des moments où elle me fatigue beaucoup. Aujourd'hui, elle est d'ailleurs particulièrement cynique, et je ne la supporte pas. Je m'apprête à décoller, je commence à m'éloigner, quand j'entends la voix de Jen faire l'écho de mes propres pensées quelques minutes auparavant.

\begin{itemize}
\item Oh merde. Elle a bougé.
\item Jen, dis-je. Tu hallucines. J'ai cru que cette tête bougeait aussi, tout à l'heure, mais je t'assure que les têtes en bocal ne boug…
\end{itemize}

Je n'ai pas le temps de finir ma phrase. Un claquement sec se fait entendre. Je me retourne pour voir que le verre du bocal contenant la tête s'est fendillé, et que la tête, qui a maintenant les yeux grands ouverts, semble avoir commencé à se dissoudre dans le liquide.

\begin{itemize}
\item Jen, qu'est-ce qui se passe ? Qu'es-ce que tu as fait à cette tête ?
\item J'ai rien fait, Pen ! Le truc a ouvert les yeux d'un coup et le verre a claqué à ce moment-là. Je…
\end{itemize}

Une explosion se fait entendre et tout devient noir.

\bnw{Et maintenant}
\item Rendez-vous au paragraphe \go{tout-est-noir}
\enw

\section{tout-est-noir}

Tout est noir. La douleur est terrible.

Le visage me brûle, j'ai mal dans tous mes membres. Comme des centaines de milliers de petits poignards qui me transpercent la peau et les os. Je souffre.

Progressivement, j'arrive à distinguer les choses autour de moi. Je vois un corps ensanglanté à côté de moi, étendu au milieu des éclats de verre d'épaisseur impressionnante. Si nous avons été frappées par cela, nul doute que nous ne sommes pas belles à voir.

\begin{itemize}
\item Jen… tu m'entends ? articulé-je péniblement.
\end{itemize}

Jen remue faiblement à côté de moi. Je remue un bras pour le tendre vers mon amie. La douleur qui me transperce aussitôt est insoutenable. Je manque de retomber dans l'inconscience, mais je m'accroche.

Et soudain, la douleur reflue, je me sens mieux, je vois plus clair autour de moi.

Les endorphines, c'est puissant. On a vu ça en cours. Mais je ne pensais pas que c'était puissant à ce point là.

Je profite de cet instant de répit pour me remettre debout. Jen remue faiblement puis plus fermement, et elle se remet debout elle aussi. Elle a le visage et le torse en sang, une vilaine balafre lui creuse un gigantesque sourire macabre autour de la bouche. Un peu comme le joker dans Batman.

Je me mets en route vers la sortie. Il nous faut des secours.

\begin{itemize}
\item À l'aide ! Il y a quelqu'un ?
\end{itemize}

Je titube vers la sortie de secours. À chaque pas, j'ai moins mal. Je me sens de moins en moins sensible à la douleur. Une étrange sensation de puissance monte en moi. Un sourire se dessine sur mes lèvres. Et en même temps, j'entends Jen ricaner à ma droite, d'un rire de soulagement. Ouf, si on est debout et qu'on ne souffre plus, ce n'est pas si grave.

Un homme entre dans la pièce devant moi. Je reconnais monsieur Tartempion, le concierge, que j'ai croisé souvent pendant les pauses. Il ouvre de grands yeux surpris en m'apercevant.

J'ouvre la bouche pour lui demander de l'aide, mais là, curieusement, je me mets en colère. Une colère pure, de la fureur brûlante. Il doit payer. Cet innocent concierge doit payer pour tous les autres.

Je pousse un hurlement de fureur mêlé d'un grognement coléreux. Je me jette sur le pauvre monsieur Tartempion, qui se laisse surprendre, et je le mords à l'épaule de toutes mes forces. Jen se joint à moi. Je suis furieuse contre lui. Nous le jetons à terre. À coups de mâchoires et de mouvements de tête, nous nous partageons des parts de butin. Une part de moi fait mine de protester, disant quelque part que ce n'est pas très logique, que les gens civilisés n'en attaquent pas d'autres sans une bonne raison… et pas avec les dents, a priori. mais je suis tellement soulagée d'avoir fait payer monsieur Tartempion que cette dernière étincelle de lucidité m'échappe et disparaît alors que je souris à Jen.

Je tords le cou à la petite voix qui me murmure que tout cela n'est pas COMPLÈTEMENT normal, puis je pars à la suite de Jen pour chercher un peu plus de nourriture.

\go{chercher-de-la-nourriture}

\section{chercher-de-la-nourriture}

Miam miam. J'ai faim. J'ai envie de faire souffrir des humains.

Les couloirs à côté du musée des horreurs sont déserts. Je me tourne vers Jen pour lui dire que nous allons continuer vers le bâtiment principal. Mais quand je parle, tout ce que j'arrive à émettre comme bruit, c'est :

\begin{itemize}
\item Hreu greumf grr aaah uuuh.
\end{itemize}

Jen, quant à elle, me répond :

\begin{itemize}
\item Grra aaah uuh rrrhhaa haa huu.
\end{itemize}

Je crois qu'on s'est comprises. Nous nous éloignons du musée des horreurs, en boitillant légèrement et en laissant une impressionnante traînée de sang derrière nous, vers le bâtiment où nous entendons que plein de vie palpite dans l'attente de nos claquements de dents.

Nous poussons une porte coupe-feu. De l'autre côté, nous croisons deux étudiants pleins de vie. Je leur souris. Je les déteste, mais je suis contente de les voir.

Je lis la peur dans leurs yeux. Ils esquissent un mouvement de recul, mais je suis plus rapide. Enfin, j'essaie. J'ai du mal à coordonner mes mouvements. Je m'emmêle légèrement les pieds et je trébuche, et je tombe à la renverse dans un gracieux « floc » humide. Je ne suis peut-être pas dans un état aussi \textit{normal} que cela après tout.

\bnw{Étalée par terre, que vais-je faire maintenant ?}
\item Tout cela est grave. Je tends la main vers eux pour leur demander de l'aide. \go{demander-de-laide}
\item Je vais me relever, et je vais les attraper pour les manger, pardi. \go{manger-les-etudiants}
\enw

\section{demander-de-laide}

\begin{itemize}
\item Hurr rgrr rrahh rrhuuh.
\end{itemize}

C'est tout ce que j'arrive à dire. Je crois que la communication va être compliquée avec les humains.

Mes deux proies et leurs cerveaux si appétissants se sauvent en courant. Attendez ! Je ne veux pas… en fait si, je veux. Je veux vous mordre jusqu'au sang, vous dévorer, vous faire payer.

\textit{Mais… vous faire payer quoi ?}

Cette petite voix raisonnable est insupportable, décidément. Je me relève en poussant péniblement sur mes bras ensanglantés. Jen me dépasse pour se lancer à la suite des deux étudiants infortunés. Je me lance à sa suite. Ces deux bouts de viande ne vont pas nous échapper comme cela.

\section{manger-les-etudiants} 

Je ne vais pas laisser une chute si anodine m'empêcher de savourer un bon repas plein de viande crue encore chaude.

\textit{Mais… pourquoi de la viande crue ?}

Ils sont partis, sur leurs muscles juteux, ils fuient notre faim dévorante. Les lâches ! Restez par ici !

.... .... ....

\section{voir-les-infos}

J'allume mon PC et je me demande à Google « incident fac médecine Lyon Rockefeller ». Je tombe sur un article écrit par un petit journal local auquel je ne prête pas attention d'habitude, car il est trop friand de nouveautés non vérifiées et de faits divers macabres.

L'article que je lis a de sinistres accents de vérité, cependant.

\begin{center}
\textit{FLASH SPÉCIAL LYON}

\textit{Plusieurs agresseurs, actuellement non identifiés, ont perpétré des attaques sanglantes sur les étudiants de la faculté de médecine avenue Rockefeller, à Grange Blanche. On fait état de nombreux blessés graves et de plusieurs morts. Suivez notre Live-Tweet en direct sur @journalPipoLyonHadeubal}.
\end{center}

Merde, MERDE ! Des gens meurent et tout ce qu'ils trouvent à faire, c'est live-tweeter l'évènement ?

Je pianote sur twitter.com pour trouver le fameux live-tweet en question.

\begin{center}
\textit{RT @tartempion Shit, évitez la fac de médecine. Il y a des tarés qui mordent tout ce qui bougent… \#{}zombieLyon}
\end{center}

\begin{center}
\textit{Certains témoins disent que des forcenés mordeurs ont réussi à s'échapper de la faculté via tram et métro. Attention chez vous ! \#{}zombieLyon}
\end{center}

\begin{center}
\textit{La police empêche tout le monde d'entrer et de sortir. \#{}zombieLyon}
\end{center}

\begin{center}
\textit{RT @trucmuche Ce sont vraiment des zombies. Il mordent les gens et ça les transforme en autres zombies. D'où le hashtag \#{}zombieLyon}
\end{center}

\begin{center}
\textit{RT @emoGirl Avec ces zombies à la fac, c'est pas encore cette fois que j'aurais\footnote{N.B. J'ai fait la faute exprès. Et ça m'a coûté.} mon année. :'(}
\end{center}

Bon. En résumé, il y a des monstres qui sont apparus dans ma fac, ils mangent les gens et transmettent leur mal aux autres. Certains twittos trouvent ça marrant de les appeler des zombies… Oui, mais c'est un peu plus compliqué que ça. Les zombies, ça existe dans les films et dans les romans. Dans la vraie vie, il y a des explications pour les phénomènes de ce genre… Oui, il y a des drogues ou des toxines qui décuplent l'agressivité, et ça suffit parfois à faire peur aux gens et aux médias. Mais un étudiant en médecine est au-dessus de ça, n'est-ce pas ?

Mais… si c'était vrai ?

\bnw{Maintenant que j'ai lu tout ça, j'éteins mon PC et…}
\item Je vais me faire cette petite sieste si longuement méritée.
\item Je retourne à la fac pour essayer d'en savoir plus et avoir des nouvelles de Jen. \go{retour-a-la-fac-1}
\item Je fais mes bagages pour fuir la ville sans un regard en arrière. Premier principe du guide de survie en territoire zombie : fuir les zones urbaines. \go{fuir-la-ville}
\enw

\section{retour-a-la-fac-1}

J'empoigne mon arc, démonté dans sa sacoche, et j'emmène un sac à dos contenant une pomme, une bouteille d'eau et quelques barres de céréales. Je ne sais pas trop vers quoi je me dirige.

Je me pose dans le premier tram qui s'arrête, direction Lyon Centre. Hélas, il ne va pas très loin. Il s'arrête un arrêt plus loin, le trafic est arrêté sur toute la ligne en raison "d'un incident voyayeur grave en station grange blanche".

Zut, les zombies cessent déjà tous les transports en commun ? Je n'ai plus qu'à continuer à pied. Heureusement, ce n'est pas bien loin. En longeant les rails d'un bon pas, j'atteins bien vite l'arrêt Ambroise Paré, le dernier avant Grange Blanche.

En bas de la pente, je distingue la lumière de nombreux gyrophares. Je continue de m'approcher. Dans les ambulances, on charge des corps recouverts de draps blancs.
 
%TODO voir où j'ai parlé de l'arrêt "les alizés". C'est pas du tout à côté de la fac de l'avenue Rockefeller.

\section{fuir-la-ville}

Je n'emporte pas beaucoup d'affaires : quelques vêtements pour le voyage, quelques provisions non périssables, mon PC. J'envoie des SMS à gauche à droite pour prévenir les gens de mon départ « en week-end ». Ulysse me demande de me reposer, et de prévenir plus tôt la prochaine fois. Jen approuve. \textit{Fuis}, m'écrit-elle.

Je prends ma voiture, garée là pour les jours où je rentre chez mes parents. Le bruit du moteur qui démarre résonne un peu tristement à mes oreilles. Je quitte Lyon, je roule tout droit jusqu'à atteindre les Alpes. J'y retrouve mes parents, qui sont surpris de me voir. Je n'étais pas censée rentrer avant la fin du mois !

Quand ils allument le poste de télé et que les nouvelles de Lyon leur parviennent, ils comprennent pourquoi j'ai tenu à quitter la ville rapidement. L'épidémie zombie s'est propagée dans la ville comme une traînée de poudre, et tous les accès à la ville ont été bloqués quelques heures à peine après mon départ. Des poches de zombies se sont déclarées un peu tout autour de la ville, mais ces épidémies ont été maîtrisées, heureusement.

Plusieurs semaines se sont écoulées. Je n'ai plus eu aucune nouvelle de Jen, ni d'Ulysse… Mais au moins, je suis en sécurité.

\begin{center}
\textit{Fin}
\end{center}

% =============================================================
% Fixe
% =============================================================


% =============================================================
% Galline
% =============================================================


% =============================================================
% Fin
% =============================================================